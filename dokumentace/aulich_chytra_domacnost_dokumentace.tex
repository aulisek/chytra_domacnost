\documentclass[a4paper,titlepage]{article}

\usepackage[czech]{babel}
\usepackage{hyperref}
\usepackage{graphicx}
\usepackage{lmodern}
\usepackage{enumitem}     % required for \setlist
\usepackage[margin=35mm]{geometry}
\setlist{noitemsep}       % no space between list items (keep margin around a list)
\frenchspacing
\renewcommand{\baselinestretch}{1.5}

\setlength{\parindent}{0pt}
\setlength{\parskip}{6pt}



\title{Maturitní projekt - Informatika}
\author{Vladislav Aulich}

\begin{document}
\maketitle
\begin{center}
	\includegraphics[width=0.5\linewidth]{obrázky/kepler_logo_vectorized_pruhl200}
\end{center}

\begin{description}
	\item [Student:] Vladislav Aulich% Zde doplňte Vaše jméno ve formátu Jméno Příjmení
	\item [Třída:] 4.C % Zde doplňte Vaši třídu
	\item [Školní rok:] 2020/2021
	\item [Vedoucí práce:] Emil Miler % Zde doplňte jméno Vašeho vedoucího
\end{description}

\maketitle
\tableofcontents
\newpage

\section{Úvod}

\subsection{Zadání projektu}

Cílem práce je vytvořit chytrou domácnost, konkrétně ovládání světel pomocí
mikrokontroleru. Systém bude možné ovládat pomocí jednoduchého rozhraní
z počítače či mobilního zařízení. Systém bude libovolně rozšiřitelný o další čidla
a senzory

\section{Popis projektu}

\subsection{Použitý hardware}

\subsubsection{Kalkulace nákladů}


\end{document}
