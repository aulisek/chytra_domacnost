\documentclass[a4paper,titlepage]{article}

\usepackage[czech]{babel}
\usepackage{hyperref}
\usepackage{graphicx}
\usepackage{lmodern}
\usepackage[margin=35mm]{geometry}
\frenchspacing
\renewcommand{\baselinestretch}{1.5}

\setlength{\parindent}{0pt}
\setlength{\parskip}{6pt}

\title{Maturitní projekt - Informatika}
\author{Vladislav Aulich}

\begin{document}

\pagestyle{empty}
\begin{center}
	\begin{center}
		\includegraphics[width=0.7\linewidth]{obrázky/kepler_logo_vectorized_pruhl200}
	\end{center}
	
	\huge{MATURITNÍ PRÁCE
		\vspace{0.5 cm}
		
		Informatika
		
		\vspace{2.5 cm}
		\textbf{Zařízení pro realizaci chytré domácnosti} }  
\end{center} 

\vspace{2.5 cm}
\Large{
	Vladislav Aulich 4.C
	
	\clearpage  
	
%--------------- První stránka --------------------
\vspace*{12 cm}

Prohlašuji tímto, že jsem  práci vypracoval samostatně pod vedením Bc. Emila Milera a uvedl v seznamu literatury veškerou použitou literaturu a další zdroje včetně internetu.

Prohlašuji rovněž, že tištěná a elektronická verze této práce jsou shodné.

V Praze dne \today            

\hspace{6.5cm} \makebox[2in]{\hrulefill}

\hspace{6.5cm} \makebox[2in]{Podpis autora}    


\clearpage
%--------------- druhá stránka --------------------
\section*{Poděkování}

Touto cestou bych rád poděkoval svému vedoucímu práce Bc. Emilu Milerovi za podporu a odbornou pomoc při tvorbě projektu.
\clearpage
%--------------- Třetí stránka --------------------

\section*{Anotace}

Tento projekt se zabývá realizací chytré domácnosti. Pro tvorbu projektu byl zvolen mikrokontrolér Arduino uno a čip ESP32. Komunikace mezi zařízeními je realizovaná bezdrátově za použití otevřené rádiové frekvence 433 MHz. Součástí projektu je i výroba hardware s použitím metody 3D tisku na tiskárně Ender 3 pro. Vývoj kódu probíhal v prostředí Arduino IDE, pro trasování změn byl využit verzovací systém git.

\clearpage
%--------------- Čtvrtá stránka ---------------------

\tableofcontents
\newpage
%--------------- Pátá stránka --------------------

\section*{Úvod}

Toto téma jsem si zvolil, protože jsem chtěl blíže prozkoumat práci s platformou Arduino a ESP32. S programováním těchto zařízení jsem měl minimální zkušenosti, proto pro mě byla práce na projektu výzvou k objevování nového. Použití komunikace na rádiové frekvenci jsem zvolil z důvodu široké škály použití a velkého množství příkladů. Zároveň jsem měl doma nevyužívaný ovladač pracující s touto frekvencí.



Motivací k výběru tématu \uv{chytré domácnosti} mi bylo její čím dál větší nasazování v domácnostech a snaha vytvořit si ji po svém. Na mnohých komerčních řešeních mi totiž nevyhovoval způsob ovládání, stejně jako velký zásah do soukromí uživatelů.
\clearpage
\section{Popis projektu}

\subsection{Použitý hardware}

\subsubsection{Kalkulace nákladů}


\end{document}
